\chapter{Nachhaltige Sicherstellung der IT-Sicherheit}
\label{chapter:abschliessende_bemerkung} 
Ein Penetrationtest entspricht einer punktuellen Bestandsaufnahme der IT-Sicherheit zu einem definierten Zeitpunkt. Unter der Sicherstellung der IT-Sicherheit wird vielmehr ein fortlaufender Prozess verstanden, der regelmäßig die Sicherheitsmechanismen überprüft und validiert. Neben der Einhaltung von technischen und organisatorischen Maßnahmen und der Umsetzung der vorgestellten Gegenmaßnahmen, wird empfohlen, die offensive Sicherheitsüberprüfung in regelmäßigen Abständen, unter Beachtung vorheriger Ergebnisse der Penetrationstests, durchzuführen. Dadurch können umgesetzte Sicherheitsmaßnahmen validiert und der Detailgrad der offensiven Sicherheitsüberprüfung kontinuierlich gesteigert werden. Nur dadurch kann die IT-Sicherheit Ihrer Systeme nachhaltig sichergestellt und die Unternehmenswerte langfristig geschützt werden. 

Bitte sprechen Sie unser Team oder Ihren Ansprechpartner an. Gerne stellen wir Ihnen unsere nachhaltigen Sicherheitslösungen für Sie maßgeschneidert vor und pflegen langjährige und vertraunsvolle Geschäftsbeziehungen. 