\chapter{Definition der Risikobewertung}
\label{chapter:risikobewertung} 
In diesem Kapitel wird die Ermittlung der Risikobewertung dargestellt. Dabei ergibt sich das Risiko anhand der Formel \texttt{Risiko = Eintrittswahrscheinlichkeit x Schadenshöhe}. Sowohl für die Eintrittswahrscheinlichkeit und Schadenshöhe kann jeweils die Bewertung mit \texttt{NIEDRIG}, \texttt{MITTEL} oder \texttt{HOCH} erfolgen. Die Beschreibung der entsprechenden Stufen zur Einordnung für die Schadenshöhe und der Wahrscheinlichkeit kann aus den folgenden Auflistungen entnommen werden.

Beschreibung der Stufen für die Eintrittswahrscheinlichkeit:
\begin{description}
\item[Niedrige Eintrittswahrscheinlichkeit]\hfill \\
Hier muss etwas stehen um den Effekt sehen zu
\item[Mittlere Eintrittswahrscheinlichkeit]\hfill \\
Hier muss etwas stehen um den Effekt sehen zu
\item[Hohe Eintrittswahrscheinlichkeit]\hfill \\
Hier muss etwas stehen um den Effekt sehen zu
\end{description}

Beschreibung der Stufen für die Schadenshöhe:
\begin{description}
\item[Niedrige Schadenshöhe]\hfill \\
Hier muss etwas stehen um den Effekt sehen zu
\item[Mittlere Schadenshöhe]\hfill \\
Hier muss etwas stehen um den Effekt sehen zu
\item[Hohe Schadenshöhe]\hfill \\
Hier muss etwas stehen um den Effekt sehen zu
\end{description}

Im Zweifel bei der Zuordnung sollte der am schlimmsten anzunehmende Fall betrachtet werden und eine höhere Risikoeinstufung erfolgen. Tabelle \ref{tab:risk_calculation} zeigt das resultierende Ergebnisse anhand der Ermittlung der Eintrittswahrscheinlichkeit sowie der Schadenshöhe.

\begin{table}[ht]
    \centering
    \begin{tabular}[h]{|c||c|c|c|}
    \hline
    \textbf{\diagbox{Schadenshöhe}{Eintrittswahr-\\scheinlichkeit}} & \textbf{NIEDRIG} & \textbf{MITTEL} & \textbf{HOCH} \\
    \hline
    \hline
    \textbf{NIEDRIG} & \cellcolor{green}\texttt{NIEDRIG} & \cellcolor{green} \texttt{NIEDRIG} & \cellcolor{yellow} \texttt{MITTEL} \\
    \hline
    \textbf{MITTEL} & \cellcolor{green} \texttt{NIEDRIG} & \cellcolor{yellow} \texttt{MITTEL} & \cellcolor{red} \texttt{HOCH} \\
    \hline
    \textbf{HOCH} & \cellcolor{yellow} \texttt{MITTEL} & \cellcolor{red} \texttt{HOCH} & \cellcolor{red} \texttt{HOCH} \\
    \hline
    \end{tabular}
\caption{Beschreibung der Risikostufen und der Risikobewertung.}
\label{tab:risk_calculation}
\end{table}


