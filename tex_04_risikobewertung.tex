\chapter{Definition der Risikobewertung}
\label{chapter:risikobewertung} 
In diesem Kapitel wird die Ermittlung der Risikobewertung dargestellt. Dabei ergibt sich das Risiko anhand der Formel \texttt{Risiko = Eintrittswahrscheinlichkeit x Schadenshöhe}. Sowohl für die Eintrittswahrscheinlichkeit als auch die  Schadenshöhe kann jeweils die Bewertung mit \texttt{NIEDRIG}, \texttt{MITTEL} oder \texttt{HOCH} erfolgen. Tabelle \ref{tab:risk_calculation} zeigt dabei die Einordnung der Risikobewertung anhand der Bewertung der Schadenshöhe sowie der Eintrittswahrscheinlichkeit. Die Beschreibung der entsprechenden Stufen zur Einordnung für die Schadenshöhe und der Wahrscheinlichkeit kann aus den folgenden Auflistungen entnommen werden.

Beschreibung der Stufen für die Eintrittswahrscheinlichkeit:
\begin{description}
\item[Niedrige Eintrittswahrscheinlichkeit]\hfill \\
Die Eintrittswahrscheinlichkeit wird mit NIEDRIG bewertet, wenn die Angriffsfläche nur für wenige Benutzer oder der verwundbare Dienst nur einer limitierten Benutzeranzahl zur Verfügung steht. Die Eintrittswahrscheinlichkeit kann auch mit NIEDRIG erachtet werden, sofern für einen verwundbaren Dienst kein öffentlicher Exploit-Code verfügbar ist oder Spezialwissen oder Insider-Wissen in (undokumentierten) Nischen-Technologien benötigt wird.

\item[Mittlere Eintrittswahrscheinlichkeit]\hfill \\
Die Eintrittswahrscheinlichkeit wird mit MITTEL bewertet, wenn die Angriffsfläche eines verwundbaren Systems einer größeren (aber limitierten) Anzahl an Benutzern zur Verfügung gestellt wird. Darüber hinaus kann öffentlich ein Exploit-Code existieren, der nur unter Anpassung mit erweiterten IT-Kenntnissen eingesetzt werden kann.
 
\item[Hohe Eintrittswahrscheinlichkeit]\hfill \\
Bei einer Eintrittswahrscheinlichkeit von HOCH ist das verwundbare System für viele Benutzer (meist über das Internet) erreichbar oder es existiert öffentlicher Exploit-Code, der ohne große Modifikation auf das Zielsystem angewendet werden kann.
\end{description}

Beschreibung der Stufen für die Schadenshöhe:
\begin{description}
\item[Niedrige Schadenshöhe]\hfill \\
Die Schadenshöhe wird mit NIEDRIG eingestuft, wenn auf das Zielsystem nur eine limitierte Anzahl an Befehlen ohne erweiterten Berechtigungen ausgeführt werden kann. Eine Schwachstelle kann ebenfalls mit NIEDRIG eingestuft werden, sofern nur wenige Daten, welche nicht unternehmenskritisch sind oder persönliche Daten betreffen aus einem System mit maliziösen Anfragen gewonnen werden können. 
\item[Mittlere Schadenshöhe]\hfill \\
Die Schadenshöhe wird mit MITTEL eingestuft, wenn ein Benutzer ohne Administrationsberechtigungen beliebige Befehle eines Systems ausführen kann und keine unternehmenskritischen Daten extrahieren kann. Die Schadenshöhe kann ebenfalls mit MITTEL bewertet werden, wenn ein Angreifer eine größere Menge persönlicher Daten oder unternehmenskritische Informationen mit maliziösen Anfragen gewonnen werden kann.

\item[Hohe Schadenshöhe]\hfill \\
Die Schadenshöhe wird mit HOCH eingestuft, wenn ein Angreifer Administrations-Berechtigungen auf einem verwundbaren System erlangt und beliebige Befehle absetzen kann. Eine Schwachstelle kann ebenfalls mit HOCH eingestuft werden, wenn sämtliche unternehmenskritische oder personenbezogene Daten in hohen Mengen aus einem Informationssystem extrahiert werden können.
\end{description}

Im Zweifel bei der Zuordnung sollte der am schlimmsten anzunehmende Fall betrachtet werden und eine höhere Risikoeinstufung erfolgen. Tabelle \ref{tab:risk_calculation} zeigt das resultierende Ergebnisse anhand der Ermittlung der Eintrittswahrscheinlichkeit sowie der Schadenshöhe.

\begin{itemize} 
	\color{red}
	\item TODO: Kategorisierung missverständilich mit nächst höherem.
\end{itemize}

\begin{table}[ht]
    \centering
    \begin{tabular}[h]{|c||c|c|c|}
    \hline
    \textbf{\diagbox{Schadenshöhe}{Eintrittswahr-\\scheinlichkeit}} & \textbf{NIEDRIG} & \textbf{MITTEL} & \textbf{HOCH} \\
    \hline
    \hline
    \textbf{NIEDRIG} & \cellcolor{green}\texttt{NIEDRIG} & \cellcolor{green} \texttt{NIEDRIG} & \cellcolor{yellow} \texttt{MITTEL} \\
    \hline
    \textbf{MITTEL} & \cellcolor{green} \texttt{NIEDRIG} & \cellcolor{yellow} \texttt{MITTEL} & \cellcolor{red} \texttt{HOCH} \\
    \hline
    \textbf{HOCH} & \cellcolor{yellow} \texttt{MITTEL} & \cellcolor{red} \texttt{HOCH} & \cellcolor{red} \texttt{HOCH} \\
    \hline
    \end{tabular}
\caption{Beschreibung der Risikostufen und der Risikobewertung.}
\label{tab:risk_calculation}
\end{table}


