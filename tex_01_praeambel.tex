\chapter{Präambel}
Dieser Abschlussbericht stellt die Ergebnisse des Penetrationstests 
\begin{itemize} 
	\color{red}
	\item TODO: ganz kurz erklären um welchen Penetrationtest es sich handelt und was das zu untersuchenden Objekt ist
\end{itemize}
dar. Zu Beginn werden im \emph{Management Summary} die Ergebnisse des Penetrationtests und der Zustand der IT-Sicherheit der Zielsysteme beschrieben. Ebenso wird die Anzahl der gefundenen Schwachstellen sowie die Schwere der potenziellen Auswirkungen in zusammengefasster Form für das Management aufgezeigt. Darüber hinaus werden empfohlene Gegenmaßnahmen sowie potentiell nötige Investitionen dargestellt. Anschließend wird das Test-System, mit dem die Sicherheitsüberprüfung der Zielsysteme durchgeführt wurde, in Kapitel \ref{chapter:details_pruefsystem} genauer beschrieben und die verwendete Software aufgelistet. In Kapitel \ref{chapter:risikobewertung} werden die verschiedenen Abstufungen und dazugehörigen Erklärungen zur Risikobewertung vorgestellt. Die \emph{\nameref{chapter:schwachstellenbeschreibung}} in Kapitel \ref{chapter:schwachstellenbeschreibung} bildet den Hauptteil des Abschlussberichts. Dabei folgt eine detaillierte Beschreibung zur Ausnutzung der Schwachstellen sowie der hinterlassenen digitalen Spuren. Zudem wird eine Risikoeinschätzung inklusive detaillierter Gegenmaßnahmen dargestellt. Abschließend folgt eine Bemerkung zur Wahrung der nachhaltigen IT-Sicherheit. 

Tabelle \ref{tab:contract_details} zeigt die in der Vertragsvereinbarung definierten Rahmenbedingungen des Penetrations-Tests. Für Details wird auf die gesonderte Ausführung der Vertragsvereinbarung unter der angegebenen Vertrags-Nummer verwiesen.


\begin{table}[h!]
    \centering
    \begin{tabular}[h]{|l|c|}
    \hline
    \textbf{Vertrags-Nr.} & wt22-j8ss0972 \\
    \hline
    \textbf{Überprüfungsart} & Penetrationstest \\
    \hline
    \textbf{Vorwissen} & Blackbox-Test \\
    \hline
    \textbf{Prüftiefe} & invasiver Penetrationstest\\
    \hline
    \textbf{Zielsystem(e)} & mb-reps.cool.datcom.prv (inkl. Subdomains)\\
    \hline
    \textbf{DNS-Server} & 172.16.77.1\\
    \hline
    \textbf{Überprüfungszeitraum} & 13. Januar 2022 bis 13. März 2022\\
    \hline
    \end{tabular}
\caption{Beschreibung der vertraglichen Bestimmungen}
\label{tab:contract_details}
\end{table}